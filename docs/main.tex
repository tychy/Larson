\documentclass{jsarticle}
\usepackage{amssymb,amsmath,amsthm}
\usepackage{newtxtt}
\usepackage[utf8]{inputenc}
\newcommand{\kakko}[1][]{(#1)}
\newcommand{\bx}{\bold{x}}
\newcommand{\bb}{\bold{b}}
\newcommand{\bd}{\bold{d}}
\newcommand{\pder}[2][]{\frac{\partial#1}{\partial#2}}
\newcommand{\dder}[2][]{\frac{\mathrm{d}#1}{\mathrm{d}#2}}
\newcommand{\Dder}[2][]{\frac{\mathrm{D}#1}{\mathrm{D}#2}}
\newcommand{\half}{\frac{1}{2}}
\newcommand{\hpn}{n + \half}
\newcommand{\hmn}{n - \half}
\newcommand{\hpj}{j + \half}
\newcommand{\hml}{j - \half}
\newcommand{\hpi}{i + \half}
\newcommand{\hmi}{i - \half}

\newcommand{\beq}{\begin{equation}}
\newcommand{\beql}[1]{\begin{equation}\label{#1}}
\newcommand{\eeq}{\end{equation}}
\newcommand{\eeqp}{\;\;\;.\end{equation}}
\newcommand{\eeqc}{\;\;\;,\end{equation}}
\newcommand{\xid}{x_i^2}
\newcommand{\lid}{l_i^2}
\newcommand{\aid}{a_i^2}

\renewcommand{\theequation}{\thesection.\arabic{equation}}
\makeatletter
\@addtoreset{equation}{section}
\makeatother

\date{\today}
\author{山田龍}
\title{原始星形成の1次元数値計算}
\begin{document}
\maketitle
\section{Introduction}
\subsection{星形成の概要}
\subsection{課題}
\section{Related Work}
\section{基礎理論}
\subsection{基礎方程式}
この論文では、自己重力と放射を入れた方程式を解く。
支配方程式は以下のようになる :\\
連続の式
\begin{equation}
    \pder[\rho]{t} = - \pder[\rho v_i]{x_i}    
\end{equation}
運動方程式
\begin{equation}
    \Dder[v_i]{t} = - \frac{1}{\rho}\pder[p]{x_i} - \pder[\Phi]{x_i}
\end{equation}
エネルギー方程式
\begin{equation}
    \Dder[e]{t} = - \frac{p}{\rho} \pder[v_j]{x_j} + \Gamma - \lambda
\end{equation}
\subsubsection{連続の式の導出}
\subsubsection{運動方程式の導出}
\subsubsection{エネルギー方程式の導出}
\subsection{ビリアル定理}

\subsubsection{ビリアル定理の導出}
\subsubsection{負の比熱}
\subsection{エムデン方程式}
球対称ガス球の平衡状態はエムデン方程式によって記述される。
\subsection{等温球の崩壊}
\subsection{逃走的収縮}
\subsection{重力不安定性}
中心部分の進化はどんどん速くなる。
周囲が取り残されて中心だけが逃走的に収縮する。
中心部分の大きさはジーンズ長程度。
中心部分の質量。
\subsection{放射}
\subsection{1stコアの形成}
\subsection{解離と電離の効果}
\subsection{2nd}
\section{衝撃波}
\subsection{衝撃波}
\subsubsection{ランキンユゴニオ}
\subsection{衝撃波の性質}
\subsection{エントロピージャンプ}
\subsection{衝撃波の大きさ}
\section{計算手法}
\subsection{差分方程式についての一般論}
\subsection{クーラン条件}
\begin{equation}
    \pder[u]{t} + c\pder[u]{x} = 0\label{eq:advection}
\end{equation}
波の伝播を表す線形移流方程式について考える。
移流速度を$c$として方程式は\eqref{eq:advection}のようになる。
この方程式の解は、$u = f(x -ct)$の形で得られ、
$c>0$ならば$x$の正の方向に、$c<0$ならば$x$の負の方向に伝播する解になる。
この方程式を$c>0$のときに風上差分法で差分化して数値的に解くことを考える。
上付き添字を時刻、下付き添字を座標に関するインデックスとおいて、
\begin{align}
    \frac{u^{n+1}_j - u^{n}_j}{\Delta t} = c \frac{u^n_{j} - u^n_{j-1}}{\Delta x}
\end{align}
と書ける。したがって、$u$は時間方向において
\begin{align}
    u^{n+1}_j  =  u^{n}_j+ c \Delta t\frac{u^n_{j} - u^n_{j-1}}{\Delta x}
\end{align}
と更新される。
クーラン条件と呼ばれ、Courant-Friedrichs-Lewy条件の略称としてCFL条件と書かれることもある。
これは$1$ステップの情報の伝達距離が格子幅を超えないという条件である。%todo:言い換え
4次中心差分法でのCFL条件
\subsubsection{フォン・ノイマンの安定性解析}
クーラン条件が満たされていることは、数値計算が安定であることを保障しない。
中心差分法、風上差分法において、安定性を考える。
%todo:波数空間の描画
\subsection{人工粘性}
\subsection{基礎方程式の差分化}
\subsection{陰的計算}
\section{結果}
\section{結論}
\end{document}
