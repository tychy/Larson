\documentclass{jsarticle}
\usepackage{newtxtt}
\date{\today}
\author{山田龍}
\title{テーマの位置づけ}
\begin{document}
\maketitle
\section{星形成の位置づけ}
\subsection{概略}
ガス領域から星が形成されて、主系列星へ進化する。
その中でも、原始星形成、質量降着、エンベロープの枯渇、前主系列星を経て主系列星へ進化する。
星形成領域にはRho Ophiuchi, Taurus Molecular Cloud, Orion Nebulaなどがある。
(写真はる)
%https://en.wikipedia.org/wiki/List_of_star-forming_regions_in_the_Local_GroupF
%https://www-tap.scphys.kyoto-u.ac.jp/~hosokawa/download/%E5%A4%A9II8.pdf
星形成領域はフィラメントのような構造をしていることもある。
原始星(質量が有意に増えつつある星)が形成されると、原始星への降着は数万年程度続く。
質量が1$M_\odot$に達すると、質量降着が止まる。
%表現変える
質量降着が終わると、1$M_\odot$の星は林トラックに乗ったあと、
ヘニエトラックに沿って進化する。そして、主系列星に至る。
\section{星形成の機構}
\subsection{重力不安定性}
ジーンズ不安定性の話
\subsection{自己相似収縮}
圧力と重力の比の話
\subsection{First Core}
収縮が断熱的になって形成される。
\subsection{Second Core}
\subsection{ガスの熱進化}
Sahaの式の話?
\section{何が知りたいのか}
星形成の過程は、暗く冷たいガスの中で進むので直接観測することでできない。
具体的には、崩壊の過程において内部が暴走的に収縮をするので中心の進化が外から見えない。
一般に原始星の形成には数十万年かかる。
分子雲の中の重力崩壊を数値的に計算する研究が行われてきた。
とくに一次元での計算はLarson(1969)やMasunagaInutsuka(2000)による仕事がある。
まず、その仕事を再現する。

\end{document}